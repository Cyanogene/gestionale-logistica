\documentclass[a4paper,12pt]{book}
\usepackage{fancyhdr}
\usepackage{xcolor}
\usepackage[hidelinks]{hyperref}
\newcommand{\coloredbold}[2]{\textcolor{#1}{\textbf{#2}}}
\renewcommand{\chaptername}{Capitolo}
\fancyhf{}
\fancyhead[R]{\thepage}
\renewcommand{\headrulewidth}{0pt}
\begin{document}
\title{Progetto Gestione Materiali}
\author
{
	Giacomo Marcon
	\\ Matteo Meneghetti  
	\\ Jiahua Zhou
	\\ Riccardo Bignami
}

\date{31 Maggio, 2020}
\maketitle


\newpage
\tableofcontents{}  
\chapter{Distinta base}   
\section{Tabella distinta base}
Per operare nella tabella \colorbox{yellow}{DISTINTA BASE} bisognerà premere sul tasto \textbf{AGGIUNGI NUOVA DISTINTA BASE}. Da qui si aprirà una videata per la creazione del nuovo componente.
Prima di creare la nuova distinta base, bisognerà assegnare le informazioni basi del componente:

\begin{itemize}
	\item Nome\textbf{*} 
	\item Codice\textbf{*} 
	\item Descrizione\textbf{*} 
	\item Lead Time
	\item Lead Time di sicurezza
	\item Scorta di sicurezza
	\item Lotto
	\item Periodo di copertura 
	\item Coefficiente d’utilizzo (assegnabile solo ai componenti del prodotto finito)        
\end{itemize}

\begin{flushright}
	\textbf{*}   : campi obbligatori
\end{flushright}

Se si deve interagire con un componente nella tabella \colorbox{yellow}{DISTINTA BASE}, basterà premere con il tasto destro del mouse e si aprirà un menu a tendina con varie scelte:

\begin{itemize}
	\item \textbf{Aggiungi sottonodo}
	\item \textbf{Modifica}: si apre una finestra dove si può modificare il componente.
	\item \textbf{Informazioni}: viene visualizzato un messaggio contenente le informazioni del componente.
	\item \textbf{Rimuovi}: prima di rimuovere il componente, viene chiesta una conferma. \coloredbold{red}{ATTENZIONE: QUESTA AZIONE NON PUO’ ESSERE ANNULLATA.}
	\item \textbf{Aggiungi al Catalogo}: il componente selezionato viene aggiunto al catalogo. 
	          
\end{itemize}

\bigskip
Se si preme il pulsante \textbf{Aggiungi sottonodo}, si aprirà un altro menu a tendina con altre opzioni:

\begin{itemize}
	\item \textbf{Carica da catalogo}: si apre una finestra dove è possibile selezionare un componente presente nel catalogo.
	\item \textbf{Carica semilavorato}: si carica da un file esterno una “DISTINTA BASE” diversa da quella presente.
	\item \textbf{Crea nuovo nodo}: si apre la finestra per la creazione di un nuovo componente.
	       
	          
\end{itemize}

Se non è caricato nessun componente nella tabella \colorbox{yellow}{DISTINTA BASE}, si può premere il ulsante \textbf{CARICA COMPONENTE DA CATALOGO} per creare la distinta base di un componente del catalogo.\\
\\Qui si aprirà una videata per la creazione del nuovo componente.
Prima di creare la nuova distinta base, bisognerà assegnare le informazioni basi del componente:

%\subsection{Pulsanti}
\section{Tabella catalogo}
\section{Descrizione tecnica}
\end{document}